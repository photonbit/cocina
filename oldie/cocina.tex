%% LyX 2.2.3 created this file.  For more info, see http://www.lyx.org/.
%% Do not edit unless you really know what you are doing.
\documentclass[a5paper,spanish]{article}
\renewcommand{\familydefault}{\sfdefault}
\usepackage[T1]{fontenc}
\usepackage[latin9]{inputenc}
\pagestyle{empty}
\setlength{\parindent}{0bp}
\usepackage{textcomp}

\makeatletter

%%%%%%%%%%%%%%%%%%%%%%%%%%%%%% LyX specific LaTeX commands.
\pdfpageheight\paperheight
\pdfpagewidth\paperwidth


\@ifundefined{date}{}{\date{}}
\makeatother

\usepackage{babel}
\addto\shorthandsspanish{\spanishdeactivate{~<>}}

\begin{document}\thispagestyle{empty}

\title{Cocina}

\author{Daniel Moreno Medina}

\maketitle
\thispagestyle{empty}
\newpage
\begin{flushright}
\textrm{\small{}Como lugar es donde todo lo que importa comienza,
donde se desarrollan de verdad las fiestas caseras, donde m�s se aprende
cuando se es ni�o.}\\
\par\end{flushright}{\small \par}

\begin{flushright}
\textrm{\small{}Como imperativo exhorta al amor propio, al orden afable
y a la cohesi�n personal e interrelacional.}
\par\end{flushright}{\small \par}

\newpage

\subsection*{Huevos Pochados\protect \\
}

Si los huevos, o quien los cocine, no est�n muy frescos: moldes, no
de papel, el horno a 180�C, una cucharada de agua y un huevo por molde
durante 11 minutos.\\

Para los d�as valientes, una olla grande con agua a 80�C, huevos cascados
sobre una espumadera o colador, depositados suavemente en el agua.
A los 4 minutos pescarlos con cuidado.

\newpage

\subsection*{Espera\protect \\
}

Pon el fuego suave,

cocina despacio.

No dejes que la prisa

queme tu bocado.\\

Haz una sola cosa,

una cada vez.

No dejes que la prisa

te lleve a ning�n lado.\\

Escucha atentamente,

haz caso de las palabras.

No dejes que la prisa

entienda lo no hablado.

\newpage

\subsection*{Ordena\protect \\
}

No es perdido

el tiempo

si deja

el coraz�n tranquilo

y la panza llena.\\

Del contento y color

de una cuidada

disposici�n

nace, casi sin querer,

un plato lleno

de emoci�n.

\newpage

\subsection*{Reza\protect \\
}

Di�logo desde dentro

al sonido de las sartenes,

sin camino marcado.\\

Sin esperar nada,

todo se da.

Sin palabras,

la idea se forma.\\

Salteando verdura,

sin formular pregunta,

la respuesta aparece.

\newpage

\subsection*{Salsa Holandesa\protect \\
}

Calienta la mantequilla en un cazo hasta que est� l�quida.\\

Separa la espuma que tenga y aparta.\\

En un bol de metal o cristal mezcla y bate las llemas, el agua, el
lim�n y la sal.\\

Calienta un cazo con agua y pon el bol encima.\\

Bate con varillas, mientras vas a�adiendo la mantequilla.\\

Sigue un rato.\\
\\

Sigue un poco m�s.\\
\\

Ya est� listo.

\newpage

\subsection*{\emph{Inventa}\protect \\
}

De seguir el camino

llegar�s, sin duda,

al final a tu destino.\\

De seguir las recetas

llegar�s, con suerte,

a los mismos sabores.\\

El camino invisible

del no medir

te llevar� al comprenderlos.

\newpage

\subsection*{Amasa\protect \\
}

Carne, masa, materia.

De la acci�n de las manos,

los ingredientes transmutan

y nos traen algo nuevo.\\

La masa ya conoce

qu� forma tendr� el pan.\\

Nuestra fuerza,

nuestros movimientos

al manipular,

�nicamente hacen la masa

despertar.

\newpage

\subsection*{Mira\protect \\
}

Vemos a veces

c�mo los ingredientes

se vuelven plato.\\

Que nuestra acci�n

no es necesaria

para que se lleve a cabo.\\

A veces,

~~~observar,

incluso sin nosotros,

~~es parte de la receta.

\newpage

\subsection*{Muffins Ingleses\protect \\
}

Saca la mantequilla de la nevera. Mezcla la harina, la levadura, el
az�car y por �ltimo la sal en un bol. Calienta el agua y la leche
en un cazo a fuego lento. S�calo del fuego cuando al introducir el
dedo no sientas nada. En todas partes te dir�n que eches el l�quido
poco a poco. Hazlo como te resulte m�s divertido. Sigue mezclando
y amasando con las manos hasta que casi se pueda despegar f�cilmente
de los dedos. A�ade la mantequilla y sigue amasando, golpeando y enrollando
en el mismo bol si no quieres manchar mucho. Cuando despu�s de mucho
estirar, doblar y apretar la masa sea homog�nea y se pueda alargar
sin que se rompa, ya est� lista para reposar en el mismo bol tapado
con un trapo. Divide la masa en pedazos, seis o diez es un buen n�mero.
Haz bolas y apl�stalas despu�s. Hazlos en la sart�n ya caliente pero
con el fuego bajo o en el horno a 180�C durante unos minutos hasta
que queden de color tostado por arriba, sin que se lleguen a quemar,
d�ndoles la vuelta a mitad de cocci�n.

\newpage

\subsection*{Disfruta\protect \\
}

Pruebas el guiso,

compruebas que va bien,

te felicitas.

Al terminarlo

el resultado es el que esperas,

te deleitas.\\

En cada acci�n

a cada momento

hemos de disfrutar

o encontrar (el momento)

en que sea placentero.

\newpage

\subsection*{Comparte\protect \\
}

Cada instante

en el que vivimos

no es nuestro.

Pertenece tambi�n

a todas las personas

con las que lo compartimos.\\

Por grande que sea la olla,

si el caldo es solo para ti,

siempre ser� menos.

\newpage

\subsection*{\emph{Vive}\protect \\
}

Sin olvidarte,

sin ser el centro,

una vez terminada

y masticada la comida

seguir pensando

hasta la siguiente.\\

Llenando cada hueco

como se llena un plato,

ni mucho ni poco,

delicioso y esmerado,

cuidando que llegue

cada momento

siempre en el presente.

\newpage

\subsection*{Huevos Benedict\protect \\
}

Pon unas tiras de bacon en el horno hasta que no puedas resistir el
olor en la cocina. Si te gusta crujiente espera un poco m�s y te lo
agradecer�s.\\

Abre un muffin, coloca una loncha de bacon, un huevo pochado y riega
con abundante salsa holandesa. Unas patatas fritas al lado completan
una comida para celebrar que estamos vivos.
\end{document}
